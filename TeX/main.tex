\documentclass[10pt,a4paper,twocolumn]{article}

\input{pisika.dat}

%  Editorial staff will uncomment the next line
% \input{staff.hed}


\begin{document}

%--------------------------------------------------------------------------
%  fill in the paper's title, author(s), and corresponding institutions
%--------------------------------------------------------------------------
\providecommand{\ShortAuthorList}[0]{David Beinhauer}
\title{Differences between Acetylcholinesterase Inhibitors}
\author[1,*]{David Beinhauer}
\affil[1]{Faculty of Science, Bioinformatics, Praque, Czech republic}
\affil[*]{david.beinhauer@email.cz}

\date{\dateline{}}

\begin{abstract}
\noindent
%---------------------------------------------------------------------------
%               Include abstract and keywords here
%---------------------------------------------------------------------------

The goal of this project is to compare two structures of human acetylcholinesterase
(AchE), focused on active sites. Each of the structure binds different ligand. 
The selected
ligands are inhibitors used in the treatment of the Alzheimer's disease. The idea is to
compare the active site iteractions and point out the differences between them. The
choosen structures are 4EY6 and 4EY7. The ligand in the first structures is 
galanthamine (GNT), in the second structure is dopenezil (E20). 


\DOI{} % do not delete this line
\end{abstract}

\maketitle
\thispagestyle{titlestyle}

%---------------------------------------------------------------------------
%               the main text of your paper begins here
%---------------------------------------------------------------------------


\section*{Introduction}

Alzheimer's disease (AD) is a chronic, progressive, neurodegenerative disorder of the 
brain characterized clinically by deterioration in the key symptoms of activities 
of daily living (ADLs), behavior, and cognition. Based on the 
cholinergic hypothesis\textsuperscript{\cite{bartus1982cholinergic}},
the cognitive decline in AD is a result of the deficits in central cholinergic 
neurotransmission resulting from a loss of acetylcholine (Ach).

Normally, the actions of Ach are terminated by a specific mechanism 
to keep the target cells from becoming overactivated. Acetylcholine is 
destroyed by an enzyme\textsuperscript{\cite{KOMERSOVA2005387}}, 
acetylcholinesterase (AchE), that is located in every 
Ach synapse. The defect course of enzymatic hydrolysis of 
Ach is considered to be one of the possible reasons of 
AD\textsuperscript{\cite{KOMERSOVA2005387}}. 
Cholinesterase inhibitors enhance central cholinergic 
function, and their usage remains the standard approach to the symptomatic 
treatment of AD\textsuperscript{\cite{grossberg2003cholinesterase}}.

In our work, we focus on two commonly AChE inhibitors for the symptomatic 
treatment of AD, galanthamine and dopenezil. Mainly, we focus on the binding site
interactions of these inhibitors with human AchE, and we will point out the 
differences between these interactions.



\section*{Methods}

\subsection*{Used structures}
Two molecular structures of human AchE binded with different ligands were 
used. PDB code of protein-ligand complex AchE with galanthamine 
(PDB code: GNT) is 4EY6. PDB code of complex AchE with dopenezil 
(PDB code: E20) is 4EY7. AchE is a homodimer. Each chain contains 
binding site for the selected ligands. Both structures were crystalized
using the same approach in the study presenting several crystal 
structures of AchE in complexes with drug 
ligands\textsuperscript{\cite{doi:10.1021/jm300871x}}. Resolution of 
4EY6 is 2.4\AA, resolution of 4EY7 is 2.35\AA. Based on PDB validation
report, quality of both structures are reasonably good and in all studied
metrics result above average quality.

Besides the mentioned ligands GNT and E20, the structures contains other ligands.
The structure 4EY6 contains besides GNT also PE8, NAG, EDO and NO3 ligands. 
The structure 4EY7 contains besides E20 also NAG, EDO and NO3 ligands. Presence 
of these ligands is not mentioned in the original 
study\textsuperscript{\cite{doi:10.1021/jm300871x}}. Possible explanation of 
presence of PE8, EDO and NO3 might be that these are crystallization artifacts.
The bonded oligosaccharide NAG might be result of post-translational 
modifications of the protein. As our study is focused on the interaction of GNT 
(respectively E020) and these additional ligands do not interact with our focused
protein-ligand interaction, we do not investigate these interactions more profoundly.


\subsection*{Methods and Workflow}
Acetylcholinesterase structures were investigated using the program PyMOL. First, 
we aligned the structures to verify its similarity. Then, we hid one of the 
chains (as it is homodimer), and we showed the AchE 
surface focused mainly on the binding side where the selected
ligands bind. Furthermore, we found the polar contact between the protein and 
the ligands, and we measured the distances of the bounds. 
Finally, we filter out the protein suroundings and focus only on the 
structure around the ligands.



\section*{Results}

\begin{figure}[tb]
  \centering
  \includegraphics[width=0.98\linewidth]{images/alignment.png}
  \caption{\textbf{Structure Comparision:} 
  Comparision of the structures 4EY6 (green) and 4EY7 (blue).}
  \label{fig:map_variants}
\end{figure} 



% \begin{figure}[tb]
%   \centering
%   \includegraphics[width=0.98\linewidth]{images/maze_variants.pdf}
%   \caption{\textbf{Varianty prostředí modelu:} 
%   Zobrazení variant prostředí modelu použitých v experimentech. 
%   Barevně rozlišujeme objekty v prostředí (viz. legenda).
%   \textbf{a} Varianta \texttt{var-0}, 
%   \textbf{b} varianta \texttt{var-1},
%   \textbf{c} varianta \texttt{var-2},
%   \textbf{d} varianta \texttt{var-3},
%   \textbf{e} varianta \texttt{var-4},
%   \textbf{f} varianta \texttt{var-5}.}
%   \label{fig:map_variants}
% \end{figure} 


% \begin{table}[t]
%   \centering % center-align tables within a column
%   \begin{tabular}{l p{5cm}}
%   \toprule
%   Proměnná & Popis \\
%   \midrule
%     \texttt{map\_object} & Objekt v buňce (možnosti:
%     volno, překážka, potrava, hnízdo). \\
%     \texttt{food\_pheromone} & Hladina feromonu pro sběr potravy.\\
%     \texttt{nest\_pheromone} & Hladina feromonu pro návrat do mraveniště.\\	
%   \bottomrule
%   \end{tabular}
%   \caption{Seznam promměnných jedné buňky mapy simulace.} \label{table:mapa} 
% \end{table}





% Please use pisikabst.bst. You may your own *.bib file.
\clearpage
\bibliographystyle{pisikabst}
\bibliography{bibfile}


\end{document}